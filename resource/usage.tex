\section{Usage}

To understand how \voidtest\ works you need to remember that it is meant to be used \italic{with}
CMake and on a \italic{one-file-one-test} basis. Technically speaking nothing stops you from
compiling it manually and putting every test case into a single translation unit but by doing so
you would be stepping into a territory not covered by tests, which is not recommended. \par
Building \voidtest\ is simple -- simply run the following commands in the directory of a top-level
\italic{CMakeLists.txt} file:

\begin{verbatim}
    cmake -S . -B build -D CMAKE_BUILD_TYPE=RelWithDebInfo
    cmake --build build --config RelWithDebInfo
    cd build
    ctest -C RelWithDebInfo
\end{verbatim}

\noindent If something goes wrong, please submit an issue report on Github. If not, you may delete
the \italic{build} directory and copy everything else into a subdirectory in your own project.
Here is an example of a CMake project source tree:

\begin{verbatim}
    external/
    |- void-test/
    |- CMakeLists.txt
    include/
    |- header.hpp
    |- ...
    source/
    |- source.cpp
    |- ...
    |- CMakeLists.txt
    test/
    |- test0.cpp
    |- test1.cpp
    |- ...
    CMakeLists.txt
\end{verbatim}

\noindent \voidtest\ defines an alias target (\code{void\_test::void\_test}) which can be used as
an argument to the \code{target\_link\_libraries()} command. To make it available, add the
\italic{external} subdirectory in the first call to \code{add\_subdirectory()} from the
main {CMakeLists.txt} and then \code{add\_subdirectory(void-test)} in the \italic{external}
directory itself.
